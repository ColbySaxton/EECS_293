\documentclass[12pt]{article}
\begin{document}
\begin{flushleft}
Jeremy Chan jsc126\\
Colby Saxton cas264\\
HW #6\\
EECS 293\\
BoardClass\\
February 27, 2019\newline
\section {Plain Text}

Board

	height = The number of pebbles that can vertically fit on the board
	width = The number of pebbles that can horizontally fit on the board
	pebbleMap = A map of pebbles where the key of a pebble is its unique integer value from the getPairing method

	pebbleOnCoordinates()
		input~ the x and y coordinates of the location to check if there is a pebble on
		output~ Whether or not there is a pebble located at this coordinate

		take the input x and y coordinates and put them into the getPairing routine to get the unique integer value for it
		Check to see if the unique integer value found above is a key existing in pebbleMap
		If there is a pebble in the map for this unique integer value:
			Then return that there is a pebble on these coordinates
		If there is not a pebble in the map for this unique integer value:
			Then return that there is not a pebble on these coordinates
		
	This routine returns a unique integer value from a a and y coordinate
	getPairing
		input~ an x and y coordinate

		Let thisX represent the x coordinate
		Let thisY represent the y coordinate
		output~ a value that represents a unique non-negative integer value made from the thisX and thisY values
\begin{center} HW #6 \end{center}
\end{flushleft}
\end{document}